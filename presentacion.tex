% FORMATO DEL DOCUMENTO BEAMER A PRESENTAR
\documentclass{beamer}						% Utilizacion del paquete beamer
\usepackage[spanish]{babel} 					% Para separar correctamente las palabras
\usepackage [utf8]{inputenc} 					% Cargamos el modulo de codificacion de caracteres para que acepte tildes y eñes
\usepackage{graphicx}						%Paquete para insertar graficas

\usecolortheme[RGB={40,50,150}]{structure} 		% Color del tema
\setbeamertemplate{items}[ball]
\setbeamertemplate{blocks}[rounded][shadow=true]

\mode<presentation> {
  \usetheme{Warsaw}
  \setbeamercovered{transparent}
}

\institute{
		\inst{}%
		\large {\textbf{Escuela Superior Politecnica del Litoral (ESPOL)}}\\
		\small
		Facultad de Ingenieria en Electricidad y Computacion (FIEC)\\
		Lenguajes de programacion
		\and
		\inst{}%
		\large {\textbf{Integrantes}}\\
		\small
		\parbox{6cm}{% Se crea una caja de texto para los integrantes
			\begin{itemize}
				 \item Victor Rodriguez
				 \item Marlon Loayza
				 \item Carlos Ramirez
			\end{itemize}}
		\parbox{1cm}{\flushright \includegraphics[scale=0.15]{./images/logo.jpg}}%Se crea una caja de texto para la imagen
		
}
%\titlegraphic[abrev.]{figura}
%\usetheme{Antibes}
\begin{document}

	%Diapositiva de presentacion
	\begin{frame}
		\title[Swarm Intelligente]{\LARGE{\textbf{DRIVE SAFE}}}
		\date{}								%Elimina la fecha de la presentacion
		\titlepage
		\scriptsize
		\setbeamerfont{title}{shape=\itshape,family=\rmfamily}
		\setbeamercolor{title}{fg=red!80!black,bg=blue!20!white}
		%\includegraphics[width=3cm]{./images/logo.jpg}
	\end{frame}	

	%Diapositiva de indice
	\begin{frame}
		\frametitle{Puntos a tocar}
		\setbeamercovered{invisible}		%Hace que el texto oculto siguiente no se muestre
		\tableofcontents[pausesections]
	\end{frame}

	%Diapositiva de los objetivos
	\begin{frame}	
		\section{OBJETIVOS}
		\frametitle{OBJETIVOS}
		
		\begin{itemize}
			\item Aplicar conocimientos necesario, para el desarrollo de la aplicacion en la plataforma movil Android.\only
			\item Ofrecer una mejora o innovar un recurso para dispositivos android\only
			\item Estimular el trabajo en equipo para la creacion de un software\only
			\item Familiarizarnos con la nueva tecnologia, participando en la creacion de ella\only
		\end{itemize}
	\end{frame}

	%Diapositiva acerca del proyecto
	\begin{frame}
		\section{PROPUESTA}
		\frametitle{PROPUESTA}
	\end{frame}

	%Diapositiva de bibliografia
	\begin{frame}
		\section{BIBLIOGRAFIA}
		\frametitle{BIBLIOGRAFIA}
		
		\begin{thebibliography}{99}
			\normalsize
			\bibitem{1} \textbf{Pagina de temas para BEAMER}\\ http://ciencias.udea.edu.co/enlaces/beamer.
			\bibitem{2} \textbf{Tutorial Beamer}\\ Nuestra primera presentación paso a paso - http://plagatux.es/2007/12/tutorial-beamer-nuestra-primera-presentacion-paso-a-paso/.
			\bibitem{3} \textbf{Introduccion a LaTeX}\\ http://webcache.googleusercontent.com/search?q=cache:4t-wB4JIAdQJ:gcp.fcaglp.unlp.edu.ar/integrantes:psantamaria:latex:start\\+\&cd=3\&hl=es\&ct=clnk\&gl=ec.
			\bibitem{4} \textbf{Edición de textos cientificos \LaTeX}\\http://sc.fcfm.buap.mx/portal/downloads/latex-moraborbon.pdf.
		\end{thebibliography}
		
	\end{frame}
	
\end{document}