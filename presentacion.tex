% FORMATO DEL DOCUMENTO BEAMER A PRESENTAR
\documentclass{beamer}						% Utilizacion del paquete beamer
\usepackage[spanish]{babel} 					% Para separar correctamente las palabras
\usepackage [utf8]{inputenc} 					% Cargamos el modulo de codificacion de caracteres para que acepte tildes y eñes
\usepackage{graphicx}						%Paquete para insertar graficas

\usecolortheme[RGB={40,50,150}]{structure} 		% Color del tema
\setbeamertemplate{items}[ball]
\setbeamertemplate{blocks}[rounded][shadow=true]

\mode<presentation> {
  \usetheme{Warsaw}
  \setbeamercovered{transparent}
}

\institute{

	\inst{}%
	\large {\textbf{Escuela Superior Politecnica del Litoral (ESPOL)}}\\
	\small
	Facultad de Ingenieria en Electricidad y Computacion (FIEC)\\
	Lenguajes de programacion
	\and
	\inst{}%

		\large {\textbf{Integrantes}}
		\small
		\begin{center}
			\begin{itemize}
				\centering \item Víctor Rodríguez \\
				%\item Carlos Ramírez\\
				%\item Marlon Loayza\\
			\end{itemize}
		\end{center}
}
%\usetheme{Antibes}
\begin{document}

	%Diapositiva de presentacion
	\begin{frame}
		\frametitle{}
		\title[Swarm Intelligente]{NOMBRE DEL PROYECTO}	
		\titlepage

		\scriptsize
		\setbeamerfont{title}{shape=\itshape,family=\rmfamily}
		\setbeamercolor{title}{fg=red!80!black,bg=blue!20!white}
	\end{frame}	

	%Diapositiva de indice
	\begin{frame}
		\frametitle{Puntos a tocar}
		\tableofcontents[pausesections]
		
		%\section{Bibliografia}
		
	\end{frame}

	%Diapositiva de los objetivos

	\begin{frame}	
		\section{OBJETIVOS}
		\frametitle{OBJETIVOS}
	\end{frame}

	%Diapositiva acerca del proyecto

	\begin{frame}
		\section{PROPUESTA}
		\frametitle{PROPUESTA}
	\end{frame}

	%Diapositiva de bibliografia
		
	\begin{frame}
		\section{BIBLIOGRAFIA}
		\frametitle{BIBLIOGRAFIA}
	\end{frame}
\end{document}