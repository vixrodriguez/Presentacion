% FORMATO DEL DOCUMENTO BEAMER A PRESENTAR
\documentclass{beamer}						% Utilizacion del paquete beamer
\usepackage[spanish]{babel} 					% Para separar correctamente las palabras
\usepackage [utf8]{inputenc} 					% Cargamos el modulo de codificacion de caracteres para que acepte tildes y eñes
\usepackage{graphicx}						%Paquete para insertar graficas
\usepackage{ragged2e}

\usecolortheme[RGB={40,50,150}]{structure} 		% Color del tema
\setbeamertemplate{items}[ball]
\setbeamertemplate{blocks}[rounded][shadow=true]

\mode<presentation> {
  \usetheme{Warsaw}
  \setbeamercovered{transparent}
}

\institute{
		\inst{}%
		\large {\textbf{Escuela Superior Politecnica del Litoral (ESPOL)}}\\
		\small
		Facultad de Ingenieria en Electricidad y Computacion (FIEC)\\
		Lenguajes de programacion
		\and
		\inst{}%
		\large {\textbf{Integrantes}}\\
		\small
		\parbox{6cm}{% Se crea una caja de texto para los integrantes
			\begin{itemize}
				 \item Victor Rodriguez
				 \item Marlon Loayza
				 \item Carlos Ramirez
			\end{itemize}}
		\parbox{1cm}{\flushright \includegraphics[scale=0.15]{./images/logo.jpg}}%Se crea una caja de texto para la imagen
		
}
%\titlegraphic[abrev.]{figura}
%\usetheme{Antibes}
\begin{document}

	%Diapositiva de presentacion
	\begin{frame}
		\title[Swarm Intelligente]{\LARGE{\textbf{DRIVE SAFE}}}
		\date{}								%Elimina la fecha de la presentacion
		\titlepage
		\scriptsize
		\setbeamerfont{title}{shape=\itshape,family=\rmfamily}
		\setbeamercolor{title}{fg=red!80!black,bg=blue!20!white}
		%\includegraphics[width=3cm]{./images/logo.jpg}
	\end{frame}	

	%Diapositiva de indice
	\begin{frame}
		\frametitle{Puntos a tocar}
		\setbeamercovered{invisible}		%Hace que el texto oculto siguiente no se muestre
		\tableofcontents[pausesections]
	\end{frame}

	%Diapositiva de los objetivos
	\begin{frame}	
		\section{OBJETIVOS}
		\frametitle{OBJETIVOS}
		
		\begin{itemize}
			\item Aplicar conocimientos necesario, para el desarrollo de la aplicacion en la plataforma movil Android.\only
			\item Ofrecer una mejora o innovar un recurso para dispositivos android\only
			\item Estimular el trabajo en equipo para la creacion de un software\only
			\item Familiarizarnos con la nueva tecnologia, participando en la creacion de ella\only
		\end{itemize}
	\end{frame}

	%Diapositiva acerca del proyecto
	\begin{frame}[allowframebreaks]
		\section{PROPUESTA}
		\frametitle{PROPUESTA}
			\Large{
			Con  las nuevas reformas en el reglamento de tránsito y las nuevas forma de sanción que 
			ha implementado la comisión de transito nacional, han surgido muchas irregularidades e inconformidades
			por esta forma de multa por sensores de velocidad, no obstante sin quitarle merito a estas reformas por
			 el decremento de los accidentes de tránsito.\\
			 \includegraphics[width=4cm]{./images/logo-cte.jpg}

			 Hemos decidido implementar una aplicación móvil para android que nos ayudara como sistema preventivo a 
			no exceder estos límites de velocidad basados en el reglamento de la comisión de transito de acuerdo a 
			la zona en donde transitemos con nuestro vehículo,  lanzando una alarma en caso de estar a punto de rebasar estos límites de velocidad y así evitar multas.\\
			
			Pero en caso contrario ignoramos esta alarma y rebasamos estos límites por alguna emergencia, conducir sin 
			las debidas precauciones, contaremos con un registro de lugares donde hemos rebasado estos límites para así 
			llevar un control y tener un aproximado de las multas que podemos tener en nuestro haber, y así tener un 
			respaldo al momento de pagar nuestras multas no ser sorprendidos por multas injustas como los casos que se han 
			presentado desde que se adopto este nuevo sistema.\\
			
			En resumen nuestra aplicación nos ayudara a prevenir excesos en la velocidad al conducir, así podremos evitar
 			accidentes y multas, junto a un control si llegamos a cometer algún exceso para el momento de pagar la multa tener
 			un soporte y saber donde y cuando cometimos la infracción.}
			
	\end{frame}

	%Diapositiva de bibliografia
	\begin{frame}
		\section{BIBLIOGRAFIA}
		\frametitle{BIBLIOGRAFIA}
		
		\begin{thebibliography}{99}
			\normalsize
			\bibitem{1} \textbf{Pagina de temas para BEAMER}\\ http://ciencias.udea.edu.co/enlaces/beamer.
			\bibitem{2} \textbf{Tutorial Beamer}\\ Nuestra primera presentación paso a paso - http://plagatux.es/2007/12/tutorial-beamer-nuestra-primera-presentacion-paso-a-paso/.
			\bibitem{3} \textbf{Introduccion a LaTeX}\\ http://webcache.googleusercontent.com/search?q=cache:4t-wB4JIAdQJ:gcp.fcaglp.unlp.edu.ar/integrantes:psantamaria:latex:start\\+\&cd=3\&hl=es\&ct=clnk\&gl=ec.
			\bibitem{4} \textbf{Edición de textos cientificos \LaTeX}\\http://sc.fcfm.buap.mx/portal/downloads/latex-moraborbon.pdf.
		\end{thebibliography}
		
	\end{frame}
	
\end{document}